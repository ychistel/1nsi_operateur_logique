\documentclass[9pt]{beamer}

\usepackage[utf8]{inputenc}
\RequirePackage[francais]{babel}
%\usepackage{url}
%\usepackage{etex}
%\usepackage{enumitem}
%\usepackage{multicol}
\usepackage{xcolor}
%\usepackage{bbm}
%\usepackage{amsmath,amsthm,amssymb}
%\usepackage[official]{eurosym}
%\usepackage{pifont}
%\usepackage{exercise}
%\usepackage{graphics}
%\usepackage{array,multirow,makecell}
%\usepackage{verbatim}
%\usepackage[dvipsnames]{pstricks}
\usepackage{pstricks-add,pst-plot,pst-text,pst-tree,pst-eps,pst-fill,pst-node,pst-math,pst-blur,pst-func}
%\usepackage{pgf,tikz}
%\usepackage{tipfr}
%\usepackage{thmbox}
%\usepackage{calc}
%\usepackage{ifthen}
%\usepackage{pdfpages}
%\usepackage{colortbl}
%\usepackage{sagetex}
%\usetikzlibrary{arrows,patterns}
%\input tabvar
%\usepackage{tkz-tab}
%\usepackage{listings}
%\usepackage[np]{numprint}
%\usepackage{fancybox,fancyhdr}
%\usepackage{thmtools}
%\usepackage{bclogo}
%\usepackage{lastpage}

\usepackage{tabularx}
\usepackage{array,multirow,makecell}
\usetheme{Madrid}
%\usetheme{Bergen}
\usecolortheme{beaver}
 
%Information to be included in the title page:
\title{Système de numération}
\subtitle{Portes logiques - Fonctions logiques}
\author{Yannick CHISTEL}
\institute{Lycée Dumont d'Urville - CAEN}
\date{\today}
 
%----------------------------------------------------------------------------------------------- 
% 							Commandes Tableaux
%-----------------------------------------------------------------------------------------------
\setcellgapes{1pt}
\makegapedcells
\newcolumntype{R}[1]{>{\raggedleft\arraybackslash }b{#1}}
\newcolumntype{L}[1]{>{\raggedright\arraybackslash }b{#1}}
\newcolumntype{C}[1]{>{\centering\arraybackslash }b{#1}}


\newcounter{num}
\setcounter{num}{0}
 
\begin{document}
 
\frame{\titlepage}

\begin{frame}
\frametitle{Les transistors}

\begin{block}{Présentation}
Dans le processeur et d'autres composants électroniques, on a des transistors qui sont des semi-conducteurs. Ils ont la particularité de laisser passer ou non le courant électrique. 

Il existe deux types de transistor : les PNP et les NPN.

En les associant, ils vont modifier les courants électriques et donc les valeurs des bits $0$ et $1$ (qui représentent le courant électrique).

\end{block}

\begin{exampleblock}{Exemple}
\begin{center}
\includegraphics[scale=0.35]{img/circuitsTransistors.eps}
\end{center}
\end{exampleblock}
\end{frame}


\begin{frame}
\frametitle{La porte logique NOT}

\begin{block}{Définition}
La porte NOT a un seul bit d'entrée et un seul bit de sortie.
\begin{itemize}
\item Si le bit d'entrée vaut $1$, alors il vaut $0$ en sortie.
\item Si le bit d'entrée vaut $0$, alors il vaut $1$ en sortie.
\end{itemize}
On donne les symbolisations de la porte NOT et la table logique:\medskip

\begin{minipage}{4cm}
\begin{pspicture}(-0.5,0)(2.5,2)
\pspolygon(0.5,0.5)(1.5,1)(0.5,1.5)
\pscircle(1.6,1){0.1}
\psline(0,1)(0.5,1)
\psline(1.7,1)(2,1)
\uput[l](0,1){\text{P}}
\uput[r](2,1){\text{Q}}
\end{pspicture}
\end{minipage}\hfill
\begin{minipage}{4cm}
\begin{pspicture}(-0.5,0)(2.5,2)
\pspolygon(0.5,0.5)(1.5,0.5)(1.5,1.5)(0.5,1.5)
%\pscircle(1.6,1){0.1}
\psline(0,1)(0.5,1)
\psline(1.5,1)(2,1)
\psline(1.5,1.2)(1.7,1)
\uput[l](0,1){\text{P}}
\uput[r](2,1){\text{Q}}
\uput[l](1.25,1){\text{1}}
\end{pspicture}
\end{minipage}\hfill
\begin{minipage}[t]{4cm}
\begin{tabular}{*{2}{|C{0.8cm}}|}\hline
P & Q\\\hline
$0$ & $1$\\\hline
$1$ & $0$\\\hline
\end{tabular}
\end{minipage}
\end{block}
\end{frame}

\begin{frame}
\frametitle{La porte logique ET}

\begin{block}{Définition}
La porte ET a 2 bits en entrée et un seul bit de sortie.
\begin{itemize}
\item Si les 2 bits d'entrée valent $1$, alors le bit de sortie vaut $1$.
\item Si un bit d'entrée ou les 2 valent $0$, alors le bit de sortie vaut $0$.
\end{itemize}
On donne les symbolisations de la porte ET et la table logique:\medskip

\begin{minipage}{3.5cm}
\begin{pspicture}(-0.5,0)(2.5,2)
%\pspolygon(0.5,0.5)(1.5,1)(0.5,1.5)
\qbezier(0.5,0.5)(1.5,0.6)(1.5,1)
\qbezier(1.5,1)(1.5,1.4)(0.5,1.5)
\psline(0.5,0.5)(0.5,1.5)
%\pscircle(1.6,1){0.1}
\psline(0,0.7)(0.5,0.7)
\psline(0,1.3)(0.5,1.3)
\psline(1.5,1)(2,1)
\uput[l](0,0.7){\text{Q}}
\uput[l](0,1.3){\text{P}}
\uput[r](2,1){\text{R}}
\end{pspicture}
\end{minipage}\hfill
\begin{minipage}{3.5cm}
\begin{pspicture}(-0.5,0)(2.5,2)
\pspolygon(0.5,0.5)(1.5,0.5)(1.5,1.5)(0.5,1.5)
%\pscircle(1.6,1){0.1}
\psline(0,0.7)(0.5,0.7)
\psline(0,1.3)(0.5,1.3)
\psline(1.5,1)(2,1)
%\psline(1.5,1.2)(1.7,1)
\uput[l](0,0.7){\text{Q}}
\uput[l](0,1.3){\text{P}}
\uput[r](2,1){\text{R}}
\uput[l](1.25,1){\text{\&}}
\end{pspicture}
\end{minipage}\hfill
\begin{minipage}[t]{5cm}
\begin{tabular}{*{2}{|C{0.8cm}}|C{1.6cm}|}\hline
P & Q & P ET Q\\\hline
$0$ & $0$ & $0$ \\\hline
$0$ & $1$ & $0$ \\\hline
$1$ & $0$ & $0$ \\\hline
$1$ & $1$ & $1$ \\\hline
\end{tabular}
\end{minipage}
\end{block}
\end{frame}

\begin{frame}
\frametitle{La porte logique OU}

\begin{block}{Définition}
La porte OU a 2 bits en entrée et un seul bit de sortie.
\begin{itemize}
\item Si les 2 bits d'entrée valent $0$, alors le bit de sortie vaut $0$.
\item Si un bit d'entrée ou les 2 valent $1$, alors le bit de sortie vaut $1$.
\end{itemize}
On donne les symbolisations de la porte OU et la table logique:\medskip

\begin{minipage}{3.5cm}
\begin{pspicture}(-0.5,0)(2.5,2)
%\pspolygon(0.5,0.5)(1.5,1)(0.5,1.5)
\qbezier(0.5,0.5)(1.5,0.6)(1.5,1)
\qbezier(1.5,1)(1.5,1.4)(0.5,1.5)
\qbezier(0.5,0.5)(0.8,0.6)(0.8,1)
\qbezier(0.8,1)(0.8,1.4)(0.5,1.5)
\psline(0,0.7)(0.7,0.7)
\psline(0,1.3)(0.7,1.3)
\psline(1.5,1)(2,1)
\uput[l](0,0.7){\text{Q}}
\uput[l](0,1.3){\text{P}}
\uput[r](2,1){\text{R}}
\end{pspicture}
\end{minipage}\hfill
\begin{minipage}{3.5cm}
\begin{pspicture}(-0.5,0)(2.5,2)
\pspolygon(0.5,0.5)(1.5,0.5)(1.5,1.5)(0.5,1.5)
\psline(0,0.7)(0.5,0.7)
\psline(0,1.3)(0.5,1.3)
\psline(1.5,1)(2,1)
\uput[l](0,0.7){\text{Q}}
\uput[l](0,1.3){\text{P}}
\uput[r](2,1){\text{R}}
\uput[l](1.5,1){$\geqslant 1$}
\end{pspicture}
\end{minipage}\hfill
\begin{minipage}[t]{5cm}
\begin{tabular}{*{2}{|C{0.8cm}}|C{1.6cm}|}\hline
P & Q & P OU Q\\\hline
$0$ & $0$ & $0$ \\\hline
$0$ & $1$ & $1$ \\\hline
$1$ & $0$ & $1$ \\\hline
$1$ & $1$ & $1$ \\\hline
\end{tabular}
\end{minipage}
\end{block}
\end{frame}

\begin{frame}
\frametitle{Fonctions booléennes}

\begin{block}{Définition}
Les portes logiques vues précédemment sont associées à des fonctions booléennes élémentaires.
\begin{itemize}
\item La porte NOT est associée à la fonction notée $\neg(x)$ ou $\overline{x}$.
\item La porte ET est associée à la fonction notée $x \wedge y$ ou $x.y$.
\item La porte OU est associée à la fonction notée $x \vee y$ ou $x+y$.
\end{itemize}

On obtient les tables de vérité :\medskip

\begin{minipage}[t]{3cm}
$$\textbf{NOT/NON}$$
\begin{tabular}{|C{0.4cm}|C{1.4cm}|}\hline
$x$ & $\neg(x)=\overline{x}$\\\hline
$0$ & $1$\\\hline
$1$ & $0$\\\hline
\end{tabular}
\end{minipage}\hfill
\begin{minipage}[t]{4cm}
$$\textbf{AND/ET}$$
\begin{tabular}{*{2}{|C{0.4cm}}|C{1.6cm}|}\hline
$x$ & $y$ & $x \wedge y=x.y$\\\hline
$0$ & $0$ & $0$ \\\hline
$0$ & $1$ & $0$ \\\hline
$1$ & $0$ & $0$ \\\hline
$1$ & $1$ & $1$ \\\hline
\end{tabular}
\end{minipage}\hfill
\begin{minipage}[t]{4cm}
$$\textbf{OR/OU}$$
\begin{tabular}{*{2}{|C{0.4cm}}|C{1.8cm}|}\hline
$x$ & $y$ & $x \vee y=x+y$\\\hline
$0$ & $0$ & $0$ \\\hline
$0$ & $1$ & $1$ \\\hline
$1$ & $0$ & $1$ \\\hline
$1$ & $1$ & $1$ \\\hline
\end{tabular}
\end{minipage}
\end{block}
\end{frame}

\begin{frame}
\frametitle{Expressions booléennes}

\begin{block}{Définition}
Les fonctions booléennes élémentaires NON, ET ,OU sont aussi appelées des opérateurs booléens et permettent d'écrire des expressions booléennes plus complexes.
\end{block}

\begin{exampleblock}{Exemple}
Soit $x$ et $y$ deux variables booléeennes. On souhaite déterminer la table de vérité de l'expression booléenne $\neg(x \vee y)$.

On écrit la table de vérité de chaque variable puis de l'expression:

\begin{center}
\begin{tabular}{*{2}{|C{0.6cm}}|C{1.2cm}|C{2.4cm}|}\hline
$x$ & $y$ & $x \vee y$ & $\neg(x \vee y)=\overline{x+y}$\\\hline
$0$ & $0$ & $0$ & $1$ \\\hline
$0$ & $1$ & $1$ & $0$\\\hline
$1$ & $0$ & $1$ & $0$\\\hline
$1$ & $1$ & $1$ & $0$\\\hline
\end{tabular}
\end{center}
\end{exampleblock}
\end{frame}

\begin{frame}
\frametitle{L'opérateur OU EXCLUSIF}

\begin{block}{Définition}
L'opérateur booléen (ou fonction booléenne) \textbf{ou} exclusif prend deux valeurs en entrée et donne en sortie:
\begin{itemize}
\item $0$ si les deux valeurs entrées sont égales
\item $1$ si les deux valeurs en entrée sont différentes
\end{itemize}
Cet opérateur booléen se note par $\oplus$ et se dit \textbf{XOR}.\medskip

On donne les symbolisations de la porte XOR et la table logique:\medskip

\begin{minipage}{3.5cm}
\begin{pspicture}(-0.5,0)(2.5,2)
%\pspolygon(0.5,0.5)(1.5,1)(0.5,1.5)
\qbezier(0.6,0.5)(1.5,0.6)(1.5,1)
\qbezier(1.5,1)(1.5,1.4)(0.6,1.5)
\qbezier(0.6,0.5)(0.9,0.6)(0.9,1)
\qbezier(0.9,1)(0.9,1.4)(0.6,1.5)
\qbezier(0.5,0.5)(0.8,0.6)(0.8,1)
\qbezier(0.8,1)(0.8,1.4)(0.5,1.5)
\psline(0,0.7)(0.7,0.7)
\psline(0,1.3)(0.7,1.3)
\psline(1.5,1)(2,1)
\uput[l](0,0.7){\text{Q}}
\uput[l](0,1.3){\text{P}}
\uput[r](2,1){\text{R}}
\end{pspicture}
\end{minipage}\hfill
\begin{minipage}{3.5cm}
\begin{pspicture}(-0.5,0)(2.5,2)
\pspolygon(0.5,0.5)(1.5,0.5)(1.5,1.5)(0.5,1.5)
\psline(0,0.7)(0.5,0.7)
\psline(0,1.3)(0.5,1.3)
\psline(1.5,1)(2,1)
\uput[l](0,0.7){\text{Q}}
\uput[l](0,1.3){\text{P}}
\uput[r](2,1){\text{R}}
\uput[l](1.5,1){$= 1$}
\end{pspicture}
\end{minipage}\hfill
\begin{minipage}[t]{5cm}
\begin{tabular}{*{2}{|C{0.8cm}}|C{1.6cm}|}\hline
P & Q & P $\oplus$ Q\\\hline
$0$ & $0$ & $0$ \\\hline
$0$ & $1$ & $1$ \\\hline
$1$ & $0$ & $1$ \\\hline
$1$ & $1$ & $1$ \\\hline
\end{tabular}
\end{minipage}
\end{block}
\end{frame}

\begin{frame}
\frametitle{Algèbre booléenne}

\begin{block}{Propriétés}
Une expression booléenne peut être transformée en suivant certaines propriétés. Pour toutes variables booléennes $x$, $y$ et $z$: 
\begin{itemize}
\item Involution : $\neg(\neg x)=\overline{\overline{x}}=x$, 
\item Neutralité : $1 \wedge x=1.x=x$ et $0 \vee x=0+x=x$, 
\item Élément absorbant : $0 \wedge x=0.x=0$, $1 \vee x=1+x=1$, 
\item Complément : $x \wedge \neg x = x.\overline{x} = 0$ et $x \vee \neg x = x + \overline{x} =1$ 
\item La commutativité : $x \wedge y = y \wedge x$ et $x \vee y = y \vee x$
\item La distributivité du ET sur le OU : $x \wedge (y \vee z)= (x \wedge y) \vee (x \wedge z)$
\item La distributivité du OU sur le ET : $x \vee (y \wedge z)= (x \vee y) \wedge (x \vee z)$
\end{itemize}
\end{block}

\begin{block}{Lois de Morgan}
Soit $x$ et $y$ deux variables booléennes:
\begin{itemize}
\item $\neg(x \wedge y) = \neg x \vee \neg y$
\item $\neg(x \vee y) = \neg x \wedge \neg y$
\end{itemize}
\end{block}

\end{frame}


\begin{frame}
\frametitle{Exercices}

\begin{block}{Exercice 1}
Soit $x$ et $y$ deux variables booléennes. 
\begin{enumerate}
\item Donner la table de vérité de l'expression booléenne $ (\neg x \wedge y) \vee x$.
\item Donner une expression simplifiée équivalente.
\end{enumerate}
\end{block}

%\begin{exampleblock}{Solution}
%\begin{enumerate}
%\item La table de vérité de l'expression booléenne  $ (\neg x \wedge y) \vee x$ est :
%\begin{center}
%\begin{tabular}{*{3}{|C{0.6cm}}|C{1.2cm}|C{2.8cm}|}\hline
%$x$ & $y$ & $\neg x$&$\neg x \wedge y$ & $(\neg x \wedge y) \vee x=\overline{x}y+x$\\\hline
%$0$ & $0$ & $1$ & $0$ & $0$ \\\hline
%$0$ & $1$ & $1$ & $1$ & $1$\\\hline
%$1$ & $0$ & $0$ & $0$ & $1$\\\hline
%$1$ & $1$ & $0$ & $0$ & $1$\\\hline
%\end{tabular}
%\end{center}
%\item D'après la table de vérité, l'expression $x \vee y$ est une expression booléenne équivalente à $ (\neg x \wedge y) \vee x$. 
%En utilisant la distributivité : \\
%$ (\neg x \wedge y) \vee x = (\neg x \vee x) \wedge (y \vee x) = 1 \wedge (y \vee x) = y \vee x = x \vee y$
%\end{enumerate}
%\end{exampleblock}
\end{frame}


\begin{frame}
\frametitle{Exercices}

\begin{exampleblock}{Solution}
\begin{enumerate}
\item La table de vérité de l'expression booléenne  $ (\neg x \wedge y) \vee x$ est :
\begin{center}
\begin{tabular}{*{3}{|C{0.6cm}}|C{1.2cm}|C{2.8cm}|}\hline
$x$ & $y$ & $\neg x$&$\neg x \wedge y$ & $(\neg x \wedge y) \vee x=\overline{x}y+x$\\\hline
$0$ & $0$ & $1$ & $0$ & $0$ \\\hline
$0$ & $1$ & $1$ & $1$ & $1$\\\hline
$1$ & $0$ & $0$ & $0$ & $1$\\\hline
$1$ & $1$ & $0$ & $0$ & $1$\\\hline
\end{tabular}
\end{center}
\medskip
\item D'après la table de vérité, l'expression $x \vee y$ est une expression booléenne équivalente à $ (\neg x \wedge y) \vee x$. 
En utilisant la distributivité : \\
$ (\neg x \wedge y) \vee x = (\neg x \vee x) \wedge (y \vee x) = 1 \wedge (y \vee x) = y \vee x = x \vee y$
\end{enumerate}
\end{exampleblock}
\end{frame}
\end{document}

