\documentclass[11pt,a4paper]{article}

\usepackage{style2017}
\newcounter{numexo}
\setcellgapes{1pt}

\begin{document}


\begin{NSI}
{Exercice}{Expressions booléennes}
\end{NSI}


\addtocounter{numexo}{1}
\subsection*{\Large Exercice \thenumexo }
Évaluer les expressions booléennes suivantes.
\begin{enumerate}
\item True AND True
\item NOT(False) AND False
\item True OR Not(False)
\item True AND False AND True
\item NOT(False) OR NOT(True)
\item False AND (NOT(True) OR NOT(False))
\end{enumerate}

\addtocounter{numexo}{1}
\subsection*{\Large Exercice \thenumexo }

\begin{enumerate}
\item Donner la table de vérité de la fonction logique $\textbf{NOT~}(X \textbf{~AND~} Y)$ qui se note $\neg(X \wedge Y)$.
\item Réaliser le circuit logique de cette fonction logique.
\item A-t-on $\neg(X \wedge Y)=\neg X \wedge \neg Y$ ? 
\item Écrire une fonction logique équivalente à $\neg(X \wedge Y)$ utilisant le OU.
\item Réaliser le circuit logique associé et vérifier vos résultats.
\item Déterminer la fonction logique équivalente à $\neg( X \vee Y)$
\end{enumerate}

\addtocounter{numexo}{1}
\subsection*{\Large Exercice \thenumexo }
L'objectif est de créer dans chacun des cas suivants, un circuit logique contenant 3 entrées binaires, des portes logiques \textbf{NOT}, \textbf{ET}, \textbf{OU} et une seule sortie binaire telle que:
\begin{enumerate}
\item la sortie vaut $0$ lorsque les trois entrées sont égales à $0$ et $1$ dans tous les autres cas.
\item la sortie vaut $1$ si et seulement si les trois entrées sont égales à $1$.
\item la sortie vaut $1$ si au moins deux entrées sont égales à $1$ (sinon la sortie vaut $0$).
\end{enumerate}

\addtocounter{numexo}{1}
\subsection*{\Large Exercice \thenumexo }
\begin{enumerate}
\item Créer un circuit logique avec 2 entrées binaires, la porte logique \textbf{XOR} et une seule sortie binaire.
\item Dresser la table de vérité de cette fonction logique. En quoi est-elle différente de la fonction logique \textbf{OU}?
\item Créer un circuit logique avec 2 entrées binaires, les portes logiques \textbf{NOT}, \textbf{ET}, \textbf{OU} et une seule sortie binaire qui se comporte comme la port logique \textbf{XOR}.
\end{enumerate}



\addtocounter{numexo}{1}
\subsection*{\Large Exercice \thenumexo }
\begin{enumerate}
\item \begin{enumerate}
\item Dresser la table de vérité de la fonction logique $(x \vee y) \vee z$.
\item Construire un circuit avec les portes logiques OU et vérifier votre table.
\item Le rôle des parenthèses est-il important ?
\end{enumerate} 
\item \begin{enumerate}
\item Dresser la table de vérité de la fonction logique $(x \wedge y) \wedge z$.
\item Construire un circuit avec les portes logiques ET et vérifier votre table.
\item Le rôle des parenthèses est-il important ?
\end{enumerate}
\item \begin{enumerate}
\item Dresser la table de vérité de la fonction logique $x \wedge (y \vee z)$.
\item Construire un circuit avec les portes logiques OU et ET et vérifier votre table.
\item Parmi les fonctions logiques proposées, laquelle donne la même table de vérité que $x \wedge (y \vee z)$ ?

\begin{tabular}{p{7cm}p{7cm}}
\textbf{a)} $(x \vee y) \wedge (x \vee z)$ & \textbf{b)}$(x \wedge y) \vee (x \wedge z)$\\
\end{tabular}
\item Réaliser le circuit et vérifier les tables de vérité.
\end{enumerate}
\end{enumerate}
\end{document}